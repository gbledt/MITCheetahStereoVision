%!TEX root = ../egpaper_final.tex

\section{Conclusion}

The presented vision system yielded promising results for use as a general terrain sensing solution for the MIT Cheetah robot during locomotion. Successful experiments showed its ability to accurately detect the location and dimensions of the stairs. Terrain information being provided to the robot helps the robot's path planning algorithms make better decisions when facing obstacles. Stability is improved as it anticipates contacts more accurately, rather than blindly relying on its reactive interactions with the physical world.

However, several practical limitations were noted when implementing the system. First and most importantly, the results were extremely sensitive to the camera calibration. As soon as the cameras moved slightly from the position they were calibrated in, the algorithms would return poor reconstructions. Similarly, if there was a slight lag between the frames that were taken on each camera, it would essentially act as if the cameras had moved with respect to each other since the robot is constantly moving. The stereo algorithm also relies on having enough distinct features to detect the disparity between the images. While most real world situation will likely have enough features, it is another consideration to keep in mind. The work presented is a good first step towards robust robot vision, but more work would still be needed in the future. Overall, the system was successful in allowing the robot to climb stairs reliably under the experiment conditions.